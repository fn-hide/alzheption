\chapter{PENDAHULUAN}
    \section{Latar Belakang Masalah}

    Pikun adalah kondisi alami yang tidak memiliki obat, menandakan bahwa ada tahap dalam kehidupan yang harus diterima dengan sabar dan ikhlas, karena itu merupakan bagian dari takdir yang ditetapkan oleh Allah \textit{subhanallahu wa ta'ala}. Kehidupan manusia melewati fase-fase yang tidak dapat dihindari, seperti lahir, tumbuh menjadi kuat, dan akhirnya kembali menjadi lemah di usia tua. Namun, agama Islam mengajarkan bahwa setiap fase hidup, termasuk usia tua dan kelemahan merupakan bagian dari ketentuan Allah \textit{subhanallahu wa ta'ala} yang memiliki makna mendalam. Islam memandang usia tua dan kelemahan sebagai bagian dari perjalanan hidup yang penuh hikmah. Kehidupan manusia tidak hanya tentang masa-masa kejayaan, tetapi juga kesulitan. Ketika manusia mengalami penurunan kognitif, hilang ingatan, atau menjadi sangat bergantung pada orang lain, itu merupakan bagian dari ujian yang Allah \textit{subhanallahu wa ta'ala} berikan. Ujian ini tidak hanya untuk orang yang mengalaminya, tetapi juga anggota keluarga dan orang-orang sekitar, mereka juga akan menghadapi ujian ini. Salah satu bentuk pengabdian hamba kepada Tuhannya adalah bagiamana seorang hamba menanggapi dengan kesabaran, keikhlasan, dan kasih sayang.

    Al-Qur'an sendiri menggambarkan siklus hidup manusia dengan jelas dalam Surat Al-Hajj ayat 5.

    \begin{flushright}
        \begin{RLtext}
            يٰٓاَيُّهَا النَّاسُ اِنْ كُنْتُمْ فِيْ رَيْبٍ مِّنَ الْبَعْثِ فَاِنَّا خَلَقْنٰكُمْ مِّنْ تُرَابٍ ثُمَّ مِنْ نُّطْفَةٍ ثُمَّ مِنْ عَلَقَةٍ ثُمَّ مِنْ مُّضْغَةٍ مُّخَلَّقَةٍ وَّغَيْرِ مُخَلَّقَةٍ لِّنُبَيِّنَ لَكُمْۗ وَنُقِرُّ فِى الْاَرْحَامِ مَا نَشَاۤءُ اِلٰٓى اَجَلٍ مُّسَمًّى ثُمَّ نُخْرِجُكُمْ طِفْلًا ثُمَّ لِتَبْلُغُوْٓا اَشُدَّكُمْۚ وَمِنْكُمْ مَّنْ يُّتَوَفّٰى وَمِنْكُمْ مَّنْ يُّرَدُّ اِلٰٓى اَرْذَلِ الْعُمُرِ لِكَيْلَا يَعْلَمَ مِنْۢ بَعْدِ عِلْمٍ شَيْـًٔاۗ وَتَرَى الْاَرْضَ هَامِدَةً فَاِذَآ اَنْزَلْنَا عَلَيْهَا الْمَاۤءَ اهْتَزَّتْ وَرَبَتْ وَاَنْۢبَتَتْ مِنْ كُلِّ زَوْجٍۢ بَهِيْجٍ
        \end{RLtext}
    \end{flushright}
    
    Artinya: “Wahai manusia, jika kamu meragukan (hari) kebangkitan, sesungguhnya Kami telah menciptakan (orang tua) kamu (Nabi Adam) dari tanah, kemudian (kamu sebagai keturunannya Kami ciptakan) dari setetes mani, lalu segumpal darah, lalu segumpal daging, baik kejadiannya sempurna maupun tidak sempurna, agar Kami jelaskan kepadamu (tanda kekuasaan Kami dalam penciptaan). Kami tetapkan dalam rahim apa yang Kami kehendaki sampai waktu yang sudah ditentukan. Kemudian, Kami mengeluarkanmu sebagai bayi, lalu (Kami memeliharamu) hingga kamu mencapai usia dewasa. Di antara kamu ada yang diwafatkan dan (ada pula) yang dikembalikan ke umur yang sangat tua sehingga dia tidak mengetahui lagi sesuatu yang pernah diketahuinya (pikun). Kamu lihat bumi itu kering. Jika Kami turunkan air (hujan) di atasnya, ia pun hidup dan menjadi subur serta menumbuhkan berbagai jenis (tetumbuhan) yang indah.” (QS. Al-Hajj 22:5).

    Ayat tersebut menegaskan bahwa ada di antara manusia yang dikembalikan pada usia tua, saat mereka kehilangan pengetahuan dan kemampuan yang pernah dimiliki. Ini mencerminkan proses alami penuaan, di mana fisik dan mental seseorang secara bertahap mengalami penurunan, mengingatkan kita akan keterbatasan manusia. Penurunan ini bukan hanya sebuah realitas biologis, tetapi juga menjadi cermin bagi manusia untuk menyadari bahwa segala sesuatu dalam hidup termasuk kekuatan dan pengetahuan merupakan anugerah dari Allah \textit{subhanallahu wa ta'ala} yang bisa diambil kapan saja. Ayat ini mengandung hikmah penting bahwa manusia dalam segala kekuatan dan kemampuan yang dimilikinya tetaplah makhluk yang lemah dan bergantung sepenuhnya kepada kehendak Allah \textit{subhanallahu wa ta'ala}. Selain itu, ayat tersebut memberikan penjelasan rinci terhadap fase kehidupan manusia sejak penciptaan hingga kematian, menunjukkan bahwa segala sesuatu dalam kehidupan adalah bagian dari siklus yang telah Allah \textit{subhanallahu wa ta'ala} tetapkan. Fase kelemahan dan pikun yang dialami seseorang di usia tua merupakan salah satu tanda dari kekuasaan Allah \textit{subhanallahu wa ta'ala}, yang menunjukkan bahwa manusia tidak pernah sepenuhnya mandiri dari ketetapan-Nya. Sebagaimana tanah yang kering menjadi subur kembali dengan turunnya hujan, demikian pula manusia melalui berbagai fase kehidupan dari kelemahan menuju kekuatan dan kembali lagi ke kelemahan. Siklus ini bukan hanya fenomena fisik, tetapi juga sebuah pengingat spiritual akan keterbatasan manusia di hadapan kekuasaan Allah \textit{subhanallahu wa ta'ala}. Dalam konteks ini, Rasulullah juga menegaskan bahwa meskipun banyak penyakit yang dapat disembuhkan, ada satu kondisi yang tidak memiliki obat, sebagaimana yang Rasulullah katakan dalam hadisnya:

    \begin{flushright}
        \begin{RLtext}
            حَدَّثَنَا حَفْصُ بْنُ عُمَرَ النَّمَرِيُّ حَدَّثَنَا شُعْبَةُ عَنْ زِيَادِ بْنِ عِلَاقَةَ عَنْ أُسَامَةَ بْنِ شَرِيكٍ قَالَ أَتَيْتُ النَّبِيَّ صَلَّى اللَّهُ عَلَيْهِ وَسَلَّمَ وَأَصْحَابَهُ كَأَنَّمَا عَلَى رُءُوسِهِمْ الطَّيْرُ فَسَلَّمْتُ ثُمَّ قَعَدْتُ فَجَاءَ الْأَعْرَابُ مِنْ هَا هُنَا وَهَا هُنَا فَقَالُوا يَا رَسُولَ اللَّهِ أَنَتَدَاوَى فَقَالَ تَدَاوَوْا فَإِنَّ اللَّهَ عَزَّ وَجَلَّ لَمْ يَضَعْ دَاءً إِلَّا وَضَعَ لَهُ دَوَاءً غَيْرَ دَاءٍ وَاحِدٍ الْهَرَمُ
        \end{RLtext}
    \end{flushright}

    Artinya: Telah menceritakan kepada kami [Hafsh bin Umar An Namari] telah menceritakan kepada kami [Sy'bah] dari [Ziyad bin 'Ilaqah] dari [Usamah bin Syarik] ia berkata, “Aku pernah mendatangi Nabi \textit{shalallahu 'alaihi wa sallam} dan para sahabatnya, dan seolah-olah di atas kepala mereka terdapat burung. Aku kemudian mengucapkan salam dan duduk, lalu ada seorang Arab badui datang dari arah ini dan ini, mereka lalu berkata, “Wahai Rasulullah, apakah boleh kami berobat?” Beliau menjawab: “Berobatlah, sesungguhnya Allah \textit{subhanallahu wa ta'ala} tidak menciptakan penyakit melainkan juga obatnya, kecuali satu penyakit, yaitu pikun” (HR. Abu Daud No. 3357). 

    Pikun di usia lanjut tidak dapat disembuhkan, pengobatan pikun hanya mengurangi tanda dan gejala serta mengoptimalkan kemampuan yang masih dimiliki \autocite{Sari2022}. Meskipun demikian, usaha untuk merawat dan mengobati penderita pikun adalah bentuk kebajikan yang tidak hanya bermanfaat bagi penderita, tetapi juga membawa pahala dan keberkahan bagi yang melakukannya.

    Penyebab paling umum terjadinya pikun atau demensia adalah Alzheimer \autocite{Suangga2024}. Alzheimer mempengaruhi sedikitnya 27 juta orang dan berhubungan dengan 60\% hingga 70\% dari semua kasus demensia \autocite{Ferreira2014}. Alzheimer merupakan penyakit yang terjadi pada bagian tengah otak yang menyebabkan hilangnya memori secara bertahap, gangguan kognitif, dan tekanan emosional \autocite{Uddindar2023}. Gejala awal penyakit ini meliputi kehilangan memori, kesulitan dengan tugas sehari-hari, penuruan bahasa, bahkan perubahan kepribadian \autocite{Lim2022}.

    Penyakit Alzheimer dimulai dengan penurunan ingatan secara ringan dan semakin memburuk seiring berjalannya waktu. Penyakit ini memiliki tiga tahap utama, yaitu ringan, sedang, dan berat. Namun, ada fase sebelumnya yang disebut Gangguan Kognitif Ringan (Mild Cognitif Impairment atau MCI). MCI adalah kondisi transisi antara pasien yang sehat dengan pasien penyakit Alzheimer . Individu dengan MCI akibat penyakit alzheimer mengalami masalah memori yang belum mengganggu aktivitas harian mereka. Pada tahap ini, mereka juga memiliki bukti indikator biologis (biomarker) yang menunjukkan perubahan pada otak yang terkait dengan Alzheimer \autocite{Dorado2022}.

    Setiap tiga detik, seseorang di suatu tempat mengalami demensia \autocite{Long2024}. Alzheimer's Disease International (ADI) memperkirakan bahwa lebih dari 55 juta orang di seluruh dunia saat ini menderita Alzheimer. Diperkirakan bahwa total biaya global untuk penyakit Alzheimer pada tahun 2015 mencapai USD 818 milyar, atau 1.09\% dari produk domestik bruto dunia \autocite{AlzheimersDiseaseInternational2022}. Mengingat penuaan merupakan penyebab utama demensia, jumlah penderitanya diperkirakan akan meningkat hingga 151 juta jiwa pada tahun 2050 \autocite{Prince2015}. Jumlah kematian terkait dengan demensia, termasuk penyakit Alzheimer diperkirakan mencapai 2,4 juta per tahun dan dapat meningkat menjadi 5,8 juta pada tahun 2040. Secara global, penyakit Alzheimer merupakan penyebab kematian kedua terbanyak setelah gagal jantung di kalangan lansia, karena pasien lebih rentan terhadap penyakit lain \autocite{Breijyeh2021,Liang2021}. Di Indonesia sendiri, diperkirakan terdapat sekitar 1.2 juta orang dengan demensia pada tahun 2016, yang akan meningkat menjadi 2 juta di 2030 dan 4 juta pada tahun 2050 \autocite{AlzheimersIndonesia2019}. Pada tahun 2020, kematian akibat penyakit Alzheimer di Indonesia mencapai 27.054 kasus, sekitar 1,6\% dari total kematian. Angka ini menunjukkan tingkat kematian yang disesuaikan dengan usia sebesar 17,01 per 100.000 penduduk, menempatkan Indonesia pada peringkat 106 di dunia dalam hal kematian akibat penyakit Alzheimer \autocite{WebAlzheimerDementia}. 

    Karena penyakit Alzheimer tidak dapat disembuhkan secara total, diagnosis dini sangat penting untuk menunda kerusakan perkembangan otak yang tidak dapat dipulihkan dan memperpanjang kemandirian pasien untuk jangka waktu yang lebih lama \autocite{Shamrat2023}. Kemajuan teknologi yang terus berkembang di bidang pencitraan medis yang menilai \textit{Medical Image Analysis} (MIA) merupakan salah satu bidang penelitian aktif dalam \textit{Computer Vision} dan memberikan informasi mendalam kepada dokter, peneliti, dan akademisi untuk mendorong lebih banyak penelitian dan analisis \autocite{McKhann2012}. \textit{Magnetic Resonance Imaging} (MRI) merupakan salah satu teknologi pencitraan yang sering digunakan untuk skrining penyakit Alzheimer pada tahap awal. Data yang dipindai MRI memiliki manfaat substansial karena memberikan resolusi spasial yang lebih baik dan fitur gambar terlihat lebih jelas untuk mendiagnosis penyakit \autocite{El-Assy2024}. Selain itu, MRI mampu memberikan citra beresolusi tinggi dan aman diterapkan pada organ otak karena tidak mengandung radiasi pengion \autocite{CitraR2024}. Meskipun MRI telah meningkatkan keakuratan diagnosis penyakit Alzheimer, interpretasi citra MRI oleh professional medis secara manual masih memiliki keterbatasan \autocite{Ismail2024}. Proses manual memakan waktu dan bergantung pada pengalaman dan keterampilan individu. Kesalahan manusia juga meningkatkan kemungkinan kesalahan diagnostik. Untuk mengurangi risiko tersebut, diperlukan suatu metode untuk diagnosis penyakit Alzheimer secara dini, yaitu menggunakan Computer Aided Diagnosis (CAD). CAD merupakan sistem komputer yang efektif dalam diagnosis dini terkait perkembangan penyakit yang hemat biaya dan tidak bias terhadap ketidakkonsistenan manusia \autocite{Lazli2019}.

    Penerapan CAD dalam diagnosis dini penyakit Alzheimer berdasarkan citra MRI otak telah banyak digunakan pada penelitian sebelumnya. Lin, dkk. mengimplementasikan Machine Learning (ML) untuk mengklasifikasi penyakit Alzheimer ke dalam tiga kategori yaitu Cognitively Normal (CN), MCI, dan Alzheimer's Disease (AD). Lin dkk mendapatkan akurasi yang cukup baik, yaitu 89.7\% menggunakan Artificial Neural Networks (ANNs) \autocite{Lin2023}. Penelitian Bari Antor, dkk. mengklasifikasi penyakit Alzheimer menjadi dua kategori yaitu CN dan AD dan mendapatkan akurasi yang cukup tinggi, yaitu 92\% menggunakan Support Vector Machine (SVM) \autocite{BariAntor2021}. Dalam analisis citra medis MRI, Salehi, dkk. menggunakan deep learning untuk deteksi dini penyakit Alzheimer dan mengklasifikasikan ke dalam tiga kelas, yaitu CN, MCI, dan AD menggunakan algoritma Convolutional Neural Network (CNN) dengan akurasi mencapai 99\% (Salehi et al., 2020). Selain itu, El-Assy, dkk. membandingkan akurasi klasifikasi penyakit Alzheimer ke dalam 3, 4, dan 5 kategori menggunakan algoritma CNN. Masing-masing mendapatkan akurasi yang tinggi yaitu 99.43\%, 99.57\%, dan 99.13\%  \autocite{El-Assy2024}.

    Algoritma Convolutional Neural Network (CNN) mampu mempelajari fitur citra dan memiliki kinerja yang baik dalam klasifikasi objek \autocite{Ahmad2020}. CNN merupakan salah satu perkembangan dari Neural Network (NN) yang digunakan untuk mengekstraksi fitur tingkat tinggi dari klasifikasi dan prediksi data citra serta merupakan algoritma DL yang paling banyak digunakan karena keberhasilannya yang tinggi dalam analisis dan klasifikasi data citra. CNN memiliki beberapa arsitektur seperti LeNet, AlexNet, VGGNet, ResNet, DenseNet, GoogleNet, dan masih banyak lagi. Salah satu arsitektur yang sering digunakan dalam pemrosesan data citra terutama dalam tugas-tugas seperti klasifikasi gambar, deteksi objek, dan segmentasi citra adalah GoogleNet. Selain itu, GoogleNet merupakan modifikasi arsitektur CNN yang berhasil menjadi model terbaik pada ImageNet Large Scale Visual Recognition Challenge (ILSVRC) pada tahun 2014 \autocite{Saidah2022}. Hal tersebut dibuktikan dari penelitian yang dilakukan oleh Karakaya, dkk. yang menguji 7 model CNN yaitu DenseNet121, EfficientNet, GoogleNet, MobileNet v3, ResNet101, dan ShuffleNet menggunakan transfer learning untuk klasifikasi dan segmentasi penyakit Alzheimer. GoogleNet menjadi model klasifikasi terbaik dengan akurasi mencapai 94.67\% kemudian membuat pipeline menggunakan model klasifikasi dan segmentasi, GoogleNet mampu membantu mengurangi beban ahli radiologi sehingga penyakit Alzheimer dapat dideteksi secara dini \autocite{Karakaya2022}. Ali, dkk. membandingkan model GoogleNet dengan AlexNet untuk klasifikasi tumor otak dalam citra MRI serta memanfaatkan model You Only Look Once (YOLO) v3 untuk deteksi lokasi tumor. Kedua model di latih dengan parameter yang berbeda agar mencapai akurasi dan efisiensi waktu yang lebih tinggi. Hasilnya GoogleNet mendapatkan akurasi yang lebih tinggi yaitu 97\% dibandingkan dengan AlexNet yang memperoleh akurasi 83\%. Model GoogleNet menunjukkan hasil yang lebih tepat dalam simulasi dengan sedikit kerugian (loss) \autocite{Ali2022}.
    
    GoogleNet (Inception v1) merupakan salah satu arsitektur CNN yang dikembangkan oleh tim Google, arsitektur ini terdiri dari 22 lapisan yang lebih dalam dibandingkan dengan model-model sebelumnya, tetapi tetap memiliki efisiensi komputasi yang lebih baik karena menggunakan strategi yang mengurangi jumlah parameter secara signifikan \autocite{Zhong2015}. Pada arsitektur CNN, peningkatan jumlah lapisan (layer) merupakan salah satu cara langsung untuk meningkatkan kompleksitas jaringan sehingga menghasilkan akurasi klasifikasi yang lebih tinggi. Namun, hal tersebut dapat membuat jaringan rentan terhadap overfitting. Inception module merupakan perkembangan dari GoogleNet yang dioptimalkan sehingga perubahan tingkat struktural membuat jaringan tidak terlalu rentan terhadap overfitting \autocite{Ahmed2020}. Inception v2 merupakan model yang dirancang untuk mengurangi kompleksitas jaringan konvolusi (convolution network) dengan membuat jaringan lebih lebar daripada lebih dalam. Menggunakan metode gabungan antara Inception v2 dan Faster RCNN, Alamsyah dan Fachrurrozi mengklasifikasi serta mendeteksi ujung jari dan memperoleh Tingkat akurasi sebesar 90\% \autocite{Alamsyah2019}. Inception v3 memiliki kinerja yang lebih unggul dalam akurasi, efisiensi komputasi, dan stabilitas pelatihan dibandingkan dengan versi sebelumnya. Perbaikan pada teknik factorization, regulasi, dan optimalisasi arsitektur membuat Inception v3 lebih kuat dalam mengatasi tugas pengenalan gambar dan mengatasi masalah overfitting \autocite{Lin2019}. Seperti pada penelitian X. Wang, dkk. Inception v3 mendapatkan akurasi sebesar 98.64\% untuk klasifikasi jenis bangunan kuno \autocite{Wang2022}. Meskipun arsitektur Inception v3 mampu mengatasi masalah overfitting yang sering terjadi dalam CNN, namun model ini memiliki struktur yang hirarki yang rumit serta penyesuaian sejumlah parameter besar mengakibatkan proses pelatihan cukup memakan waktu \autocite{Mujahid2022}. Oleh karena itu diperlukan pemisahan feature learning dan klasifikasi untuk masalah tersebut \autocite{Bae2020}.

    Beberapa penelitian menggunakan fitur convolutional pada arsitektur CNN sebagai feature learning, sedangkan metode klasifikasi menggunakan metode berbeda \autocite{Novitasari2022}. Enireddy, dkk. mengimplementasikan ResNet50 sebagai ektraksi fitur dan SVM sebagai pengklasifikasi untuk deteksi orang yang terinveksi virus COVID-19 menghasilkan rata-rata akurasi 97\% \autocite{Enireddy2020}. AlTahhan, dkk. menerapkan model satu jaringan dan model hybrid untuk mengklasifikasikan kelas tumor otak. Penelitian tersebut mendapatkan masing-masing akurasi sebesar 88\%, 85\%, 95\%, dan 97\% untuk model GoogleNet, AlexNet, AlexNet-SVM, dan AlexNet-KNN \autocite{AlTahhan2023}. Penelitian yang dilakukan oleh Y. Zhao, dkk. dengan mengimplementasikan model Inception v3 sebagai ekstraksi fitur yang mempelajari urutan data video sebagai pengenalan dan metode LSTM sebagai deteksi kelelahan dan kantuk menghasilkan akurasi sebesar 96.5\%.  \autocite{Zhao2020}. Kursini mengimplementasikan arsitektur Inception v3 untuk memodelkan dan mempelajari fitur video spasial dan metode LSTM digunakan untuk memvalidasi kategori video berdasarkan kumpulan data sebelumnya. Penelitian ini mendapatkan akurasi mencapai 97.4\%. \autocite{Kusrini2022}. Berdasarkan penelitian diatas, CNN arsitektur Inception v3 sebagai feature learning mampu mempelajari fitur citra dengan baik, selain itu berdasarkan penelitian yang dilakukan oleh AlTahhan, dkk. menunjukkan bahwa model hybrid mampu memaksimalkan akurasi daripada model satu jaringan \autocite{AlTahhan2023}.

    Extreme Learning Machine (ELM) banyak menarik perhatian dibidang penelitian karena merupakan algoritma machine learning yang sederhana dan cepat berdasarkan arsitektur Single-Hidden Layer Feed Forward Neural Network (SLFN) \autocite{Li2022}. Beberapa penelitian sebelumnya telah membuktikan bahwa ELM lebih unggul dibandingkan dengan ANN dan SVM dalam mengatasi masalah minimum lokal pada analisis regresi dan dapat terus dioptimalkan oleh algoritma lain untuk meningkatkan kemampuan generalisasi \autocite{Kardani2022,Yaseen2018}. ELM telah berhasil diterapkan dalam banyak tugas real-time learning untuk klasifikasi, clustering, dan regresi. J. Wang, dkk. juga menunjukkan bahwa ELM bekerja cukup baik dalam tugas pencitraan medis seperti MRI, CT, dan mammogram. \autocite{Wang2022}. Afza, dkk. melakukan klasifikasi multiclass skin lesion menggunakan metode hybrid, dimana Hybrid Whale Optimization (HWO) dan Entropy Mutual Information (EMI) digunakan sebagai best feature selection sementara ELM digunakan sebagai pengklasifikasinya. Penelitian ini menggunakan dua sumber data berbeda, yaitu HAM10000 dan ISIC2018 yang masing-masing mendapatkan akurasi sebesar 93,4\% dan 94,36\% (Afza et al., 2022). Priyadharshini, dkk. melakukan identifikasi kanker kulit melanoma menggunakan klasifikasi hybrid ELM-TLBO (Teaching Learning Based Optimization) yang menghasilkan akurasi sebesar 93.18\%. ELM merupakan SLFN yang dapat dilatih dengan cepat dan akurat (Priyadharshini et al., 2023). Meskipun banyak kelebihan, metode ELM juga memiliki kekurangan, salah satunya jumlah hidden nodes ditentukan melalui trial and error, sehingga tidak mungkin mengetahui jumlah hidden nodes yang tepat, sehingga muncul beberapa modifikasi metode ELM salah satunya yaitu dengan penambahan kernel yang biasa dikenal dengan Kernel Extreme Learning Machine (KELM) (Aisah et al., 2023).

    KELM merupakan perkembangan dari ELM yang dapat digunakan untuk masalah klasifikasi skala besar. KELM mampu beroperasi sangat cepat baik dalam fase pelatihan maupun pengujian sehingga memberikan evaluasi eksperimental yang luas dalam masalah klasifikasi skala menengah dan skala besar (Iosifidis et al., 2017). Attique, dkk melakukan klasifikasi infeksi pneumonia COVID-19 dari CT-scan menggunakan One-Class KELM dan mendapatkan akurasi sekitar 95,1\% dengan tingkat sensivitas, spesifisitas, dan presisi masing-masing 95,1\%, 95\%, dan 94\% (Attique et al., 2021). Gaspar, dkk. menggunakan metode KELM yang dioptimalkan melalui algoritma Giza Pyramids Construction (GPC) untuk klasifikasi Autism Spectrum Disorder (ASD) menghasilkan akurasi rata-rata 98,8\% (Gaspar et al., 2022).

    Berdasarkan pemaparan sebelumnya, penelitian ini akan membuat sistem klasifikasi penyakit Alzheimer menjadi 6 kategori dengan menggunakan Hybrid metode CNN dengan arsitektur Inception v3 sebagai feature learning dan KELM digunakan untuk klasifikasi penyakit Alzheimer. Sebelum melakukan feature learning menggunakan model Inception v3, penelitian ini melakukan augmentation untuk meningkatkan fitur dan menyeimbangkan dataset, kemudian KELM bekerja sebagai pengklasifikasi. Dengan memanfaatkan kekuatan kedua metode ini, penulis berharap penelitian ini dapat membangun model klasifikasi yang dapat mengklasifikasikan penyakit Alzheimer secara tepat dan efisien.

    \section{Rumusan Masalah}
    Berdasarkan penguraian masalah yang terdapat pada latar belakang di atas, memunculkan beberapa rumusan masalah yang akan diselesaikan pada penelitian ini yaitu:
    \begin{enumerate}
        \item Bagaimana hasil ekstraksi fitur data MRI menggunakan Inception v3?
        \item Bagaimana hasil optimal klasifikasi penyakit Alzheimer menggunakan metode hybrid Inception v3-KELM?
    \end{enumerate}

    \section{Tujuan Penelitian}
    Berdasarkan rumusan masalah di atas, memunculkan tujuan penelitian sebagai berikut:
    \begin{enumerate}
        \item Mengetahui hasil ekstraksi fitur data citra MRI menggunakan Inception v3.
        \item Mengetahui hasil optimal klasifikasi fitur berdasarkan ekstraksi fitur data citra MRI menggunakan metode hybrid Inception v3-KELM.
    \end{enumerate}

    \section{Manfaat Penelitian}
    Penelitian ini memiliki beberapa manfaat bagi berbagai pihak seperti yang dipaparkan berikut ini:
    \begin{enumerate}
        \item Manfaat Teoritis
        Penelitian ini diharapkan dapat menjadi referensi untuk para peneliti berikutnya dalam klasifikasi kategori penyakit Alzheimer menggunakan Convolutional Neural Network (CNN) arsitektur Interception v3 berdasarkan citra MRI otak.

        \item Manfaat Praktis
        \begin{enumerate}
            \item Bagi Penulis\\
            Meningkatkan wawasan baru bagi penulis dalam menerapkan algoritma Convolutional Neural Network (CNN) Interception v3 sebagai ekstraksi fitur sedangkan metode KELM diterapkan untuk klasifikasi citra MRI penyakit Alzheimer.

            \item Bagi Tim Medis\\
            Membantu tim medis dalam mendiagnosis penyakit Alzheimer dengan lebih mudah, cepat, dan hasil akurasi yang maksimal.
        \end{enumerate}
    \end{enumerate}

    \section{Batasan Masalah}
    Penelitian ini memerlukan batasan masalah karena mempertimbangkan ruang lingkup permasalahan yang sangat luas sehingga batasan permasalahan pada penelitian ini seperti dipaparkan sebagai berikut:
    \begin{enumerate}
        \item Metode deep learning yang diimplementasikan pada penelitian ini sebagai ekstraksi fitur adalah metode Interception v3, sedangkan metode KELM diterapkan sebagai klasifikasi kategori penyakit Alzheimer
        \item Data yang digunakan pada penelitian ini untuk mengklasifikasikan kategori penyakit Alzheimer adalah data citra MRI otak yang berasal dari Alzheimer's Disease Neuroimaging Initiative (ADNI).
        \item Kategori penyakit Alzheimer yang digunakan pada penelitian ini untuk mengklasifikasikan kategori penyakit Alzheimer adalah cognitively normal, subjective memory complaints, early mild cognitive impairment, mild cognitive impairment, late mild cognitive impairment, dan alzheimer's disease.
    \end{enumerate}

    \section{Sistematika Penulisan}
    Penelitian ini tersusun atas lima bab yang memuat seluruh isi penelitian dan diringkas pada sistematika penulisan sebagai berikut:
    \begin{enumerate}
        \item BAB I PENDAHULUAN\\
        Bab ini berisi latar belakang masalah, permasalahan, pembatasan masalah, tujuan dan manfaat penulisan, serta sistematika penulisan.
        \item BAB II TINJAUAN PUSTAKA\\
        Bab ini memaparkan tentang teori-teori yang digunakan pada penelitian ini berdasarkan jurnal, buku, dan referensi lain yang mendukung penelitian ini. Tinjauan pustaka penelitian ini memuat teori tentang penyakit Alzheimer dan citra MRI, metode pada tahap pre-processing menggunakan cropping, resize, dan CLAHE; augmentasi data; metode ekstraksi menggunakan Inception v3; metode pada tahap klasifikasi penyakit Alzheimer menggunakan KELM, dan metode pada tahap evaluasi.
        \item BAB III METODE PENELITIAN\\
        Bab ini memaparkan tentang proses memperoleh dan mengolah data untuk menyelesaikan rumusan masalah pada penelitian ini.
        \item BAB IV HASIL DAN PEMBAHASAN\\
        Bab ini memaparkan tentang hasil penelitian terkait proses yang terjadi pada klasifikasi penyakit Alzheimer dan analisis yang diperoleh.
        \item BAB V PENUTUP\\
        Bab ini memaparkan tentang hasil penelitian terkait proses yang terjadi pada klasifikasi penyakit Alzheimer dan analisis yang diperoleh.
    \end{enumerate}