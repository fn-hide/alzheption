\chapter{TINJAUAN PUSTAKA}

\section{Penyakit Alzheimer}
    Demensia adalah sindrom yang ditandai dengan penurunan kemampuan kognitif, seperti ingatan, berpikir, dan penalaran yang cukup parah hingga menggangu fungsi sehari-hari \autocite{Al-Finatunnimah2020}. Demensia bukanlah sebuah penyakit tertentu, melainkan sekelompok gejala yang terkait dengan kondisi yang mendasarinya, seperti vascular dementia, penyakit Alzheimer, \textit{front temporal dementia} atau \textit{Lewy body dementia} \autocite{Sianturi2021}.

    Penyakit Alzheimer merupakan penyebab demensia yang paling umum \autocite{Dementia2021}. Penyakit Alzheimer adalah penyakit degeneratif otak yang ditandai dengan penurunan kemampuan memori, bahasa, pemecahan masalah dan keterampilan kognitif lainnya yang mempengaruhi kemampuan seseorang untuk melakukan kegiatan sehari-hari. Penurunan ini terjadi karena sel-sel saraf (neuron) di bagian otak yang terlibat dalam fungsi kognitif telah rusak dan tidak lagi berfungsi normal \autocite{Breijyeh2021}. Orang yang mengidap Alzheimer dapat mengalami perubahan suasana hati, kepribadian, atau perilaku dan juga beberapa fungsi tubuh akibat dari kerusakan saraf. Pada akhirnya penderita dapat mengalami kematian setelah beberapa tahun karena kemampuan motoriknya sudah tidak berfungsi \autocite{Association2024}. Selain itu, faktor risiko seperti genetik, penyakit serebrovaskular, hipertensi, diabetes tipe 2, kegemukan/obesitas, kadar kolestrol tinggi, stress/depresi, kurang tidur, serta merokok menjadi penyebab dari penyakit Alzheimer \autocite{Vinicius2019}.

    \begin{figure}[H] 
        \begin{center} 
            \includegraphics[height=6cm]{img/chapter2/Otak Normal vs AD.png}
            \caption{Struktur fisiologis otak dan neuron pada (a) otak sehat dan (b) otak penderita penyakit Alzheimer} 
            \label{fig:brain-normal-ad}
            Sumber: \autocite{Breijyeh2021}
        \end{center} 
    \end{figure}

    \subsection{\textit{Cognitively Normal} (CN)}
    CN merupakan kondisi individu yang menunjukkan fungsi kognitif berada dalam rentang normal, meskipun berada dalam kelompok usia lanjut atau memiliki faktor risiko seperti riwayat keluarga dengan penyakit \textit{neuro-degenerative}. Individu yang termasuk dalam kategori CN tidak menunjukkan tanda-tanda klinis dari gangguan kognitif atau demensia seperti lupa ingatan, meskipun mungkin ada tanda-tanda awal perubahan di otak mereka yang tidak terlihat dalam kehidupan sehari-hari. Dalam citra MRI dari individu CN biasanya menunjukkan struktur otak yang masih utuh tanpa adanya atrofi yang signifikan atau penumpukan plak amiloid dan tau yang biasanya ditemukan pada pasien Alzheimer \autocite{Xie2023}. Kategori CN seperti terlihat pada Gambar \ref{fig:cn}.
    \begin{figure}[H] 
        \begin{center} 
            \includegraphics[height=6cm]{img/chapter2/CN.jpg}
            \caption{Kategori CN} 
            \label{fig:cn}
            Sumber: XXXXXXXXXXXXXX
        \end{center} 
    \end{figure}

    \subsection{\textit{Subjective Memory Complaints} (SMC)}
    SMC adalah kondisi di mana individu merasa mengalami penurunan memori atau kemampuan kognitif, meskipun pengujian \textit{neuro-psychology} formal mungkin tidak menunjukkan defisit yang signifikan. SMC dapat menjadi tanda awal perubahan \textit{neuro-degenerative}, hal ini membuat beberapa individu merasa khawatir. Melalui citra MRI, individu dengan SMC terkadang menunjukkan perubahan \textit{subtile} di otak, seperti penurunan volume di \textit{hippocampus} atau \textit{korteks entorhinal}, yang merupakan area otak yang pertama kali terkena dampak dalam penyakit Alzheimer \autocite{Dhana2022}. Kategori SMC terlihat seperti pada Gambar \ref{fig:smc}.
    \begin{figure}[H] 
        \begin{center} 
            \includegraphics[height=6cm]{img/chapter2/SMC.png}
            \caption{Kategori SMC} 
            \label{fig:smc}
            Sumber: XXXXXXXXXXXXXX
        \end{center} 
    \end{figure}

    \subsection{\textit{Early Mild Cognitive Impairment} (EMCI)}
    EMCI adalah tahap awal dari gangguan kognitif ringan yang sering kali dianggap sebagai transisi antara penuaan normal dan gangguan kognitif ringan. EMCI disebut juga sebagai tahap awal dari MCI dengan gangguan memori \textit{episodic} yang lebih ringan. Pada citra otak menggunakan MRI, individu dengan EMCI sering menunjukkan perubahan structural yang lebih jelas dibandingkan dengan mereka yang hanya mengalami penuaan normal atau SMC. Perubahan ini termasuk penurunan volume di area otak seperti \textit{hippocampus}, \textit{korteks entorhinal}, dan \textit{lobus temporal}, yang semuanya terkait dengan fungsi memori. Selain itu, terdeteksi perubahan struktur mikro pada otak dengan mengukur integritas materi putih (\textit{white matter}), yang mungkin mencerminkan proses \textit{neuro-degenerative} sedang berlangsung pada tahap awal ini \autocite{Kang2020}. Kategori EMCI terlihat seperti pada Gambar \ref{fig:emci}.
    \begin{figure}[H] 
        \begin{center} 
            \includegraphics[height=6cm]{img/chapter2/EMCI.png}
            \caption{Kategori EMCI} 
            \label{fig:emci}
            Sumber: XXXXXXXXXXXX
        \end{center} 
    \end{figure}

    \subsection{\textit{Mild Cognitive Impairment} (MCI)}
    MCI adalah tahap lanjut dari EMCI, kondisi di mana individu mengalami penurunan kognitif yang lebih besar dari yang diharapkan untuk usianya, namun belum cukup parah untuk dikategorikan sebagai demensia. Terdapat dua jenis MCI, yaitu \textit{amnestic} MCI (yang mempengaruhi memori) dan \textit{non-amnestic} MCI (yang mempengaruhi kemampuan kognitif lainnya seperti bahasa atau persepsi visual-spasial). Pada citra otak MRI, perubahan pada MCI tidak jauh berbeda dengan EMCI. Namun, terdapat penipisan \textit{korteks} di area \textit{korteks entorhinal} dan \textit{lobus temporal medial}, yang juga berhubungan dengan penurunan kognitif. Selain itu, perubahan lebih besar pada \textit{white matter} dan penurunan konektivitas fungsional antara berbagai wilayah otak dapat diamati pada individu dengan MCI \autocite{Song2021}. Kategori MCI terlihat seperti Gambar \ref{fig:mci}.
    \begin{figure}[H] 
        \begin{center} 
            \includegraphics[height=6cm]{img/chapter2/MCI.jpg}
            \caption{Kategori MCI} 
            \label{fig:mci}
            Sumber: XXXXXXXXXXXX
        \end{center} 
    \end{figure}

    \subsection{\textit{Late Mild Cognitive Impairment } (LMCI)}
    LMCI merupakan tahap lebih lanjut dari MCI, di mana penurunan fungsi kognitif menjadi lebih nyata dan mulai mengganggu beberapa aktivitas sehari-hari. LMCI sering dianggap sebagai tahap peralihan kritis menuju Alzheimer, karena pada tahap ini kerusakan neurologis dan perubahan struktural di otak lebih signifikan dibandingkan dengan tahap-tahap sebelumnya. Pada MRI, individu dengan LMCI sering menunjukkan penurunan volume otak yang lebih besar, terutama di hippocampus dan adanya kerusakan lebih lanjut pada \textit{white matter}. Penipisan \textit{korteks} juga terlihat lebih jelas \autocite{Prado2021}. Kategori LMCI seperti terlihat pada gambar \ref{fig:lmci}.
    \begin{figure}[H] 
        \begin{center} 
            \includegraphics[height=6cm]{img/chapter2/LMCI.jpg}
            \caption{Kategori LMCI} 
            \label{fig:lmci}
            Sumber: XXXXXXXXXXXXXX
        \end{center} 
    \end{figure}

    \subsection{\textit{Alzheimer's Disease} (AD)}
    AD merupakan tahap terakhir dari spektrum gangguan \textit{neuro-degenerative} di mana pasien sangat bergantung pada pengasuh atau orang lain, hal ini disebabkan pasien mengalami kesulitan besar dalam mengingat, berpikir, dan melakukan tugas-tugas dasar. Pada tahap ini, citra otak MRI menunjukkan perubahan struktural yang sangat signifikan. Penurunan volume yang luas terlihat jelas pada bagian \textit{hippocampus}, \textit{korteks entorhinal}, dan \textit{lobus temporal medial}. \textit{Korteks serebral} secara keseluruhan menjadi lebih tipis dan \textit{ventrikel} otak melebar, mencerminkan hilangnya jaringan otak. MRI juga menunjukkan penurunan integritas \textit{white matter} yang parah sehingga mengakibatkan disfungsi konektivitas antara berbagai wilayah otak. Perubahan ini menunjukkan kerusakan luas yang terjadi pada jaringan saraf dan menunjukkan tingkat \textit{neuro-degenerasi} yang sangat parah \autocite{Yamanakkanavar2020}. Kategori AD terlihat seperti pada Gambar \ref{fig:ad}.
    \begin{figure}[H] 
        \begin{center} 
            \includegraphics[height=6cm]{img/chapter2/AD.jpg}
            \caption{Kategori AD} 
            \label{fig:ad}
            Sumber: XXXXXXXXXXXX
        \end{center} 
    \end{figure}

\section{\textit{Resize}}
    
Penelitian ini melakukan resize data masukan menjadi ukuran $299\times 299$ sesuai dengan model Inception v3 untuk mengurangi waktu komputasi selama pelatihan model. Pada pengolahan gambar, \textit{resize} adalah teknik untuk mengubah ukuran gambar tanpa mengubah gambar aslinya \autocite{SyechAhmad2023}. Proses \textit{resize} atau pengubahan skala gambar dapat menambah atau mengurangi resolusi gambar target sehingga absolut data gambar disesuaikan. Penelitian ini menggunakan \textit{Interpolasi Bilinier} untuk melakukan \textit{resize} karena teknik ini memberikan kinerja kualitas gambar yang lebih baik (misalnya pengaburan tepi, diskontinuitas di tepi, dan efek papan catur) dibandingkan dengan metode \textit{nearest-neighbor}. Selain itu, \textit{interpolasi bilinier} dapat berjalan dengan waktu komputasi dan kebutuhan memori yang lebih rendah daripada \textit{resize} gambar berbasis \textit{interpolasi bicubic} \autocite{Ahn2018}.

\begin{figure}[H] 
    \begin{center} 
        \includegraphics[height=6cm]{img/chapter2/Resize.png}
        \caption{Ilustrasi \textit{Resize}} 
        \label{fig:resize}
        Sumber: \autocite{Ahn2018}
    \end{center} 
\end{figure}

Saat melakukan \textit{resize}, \textit{interpolasi bilinier} menghitung piksel baru menggunakan kombinasi linier dari empat piksel masukan berbobot. Gambar \ref{fig:resize} menunjukkan cara kerja \textit{interpolasi bilinier} untuk melakukan interpolasi piksel P dalam nilai piksel A, B, C, D yang diketahui. $\alpha$ dan $\beta$ adalah proporsi tinggi sedangkan p dan q adalah proporsi lebar, dengan perhitungan seperti pada persamaan \ref{eq:resize1}.

\begin{align}
    \label{eq:resize1}
    \alpha = \frac{h_1}{h_1 + h_2}, \beta = \frac{h_2}{h_1 + h_2}, p = \frac{w_1}{w_1 + w_2}, q = \frac{w_2}{w_1 + w_2}
\end{align}

Kemudian, piksel P dapat dihitung menggunakan persamaan \ref{eq:resize2}.

\begin{align}
    V &= q\left( \beta A + \alpha B \right) \nonumber \\
    V' &= p\left( \beta D + \alpha C \right) \nonumber \\
    \label{eq:resize2}
    P &= V + V'
\end{align}

\section{\textit{Contrast Limited Adaptive Histogram Equalization} (CLAHE)}
